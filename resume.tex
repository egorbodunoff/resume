\documentclass[letterpaper,10pt]{article}
\usepackage{preambule}


\begin{document}

  

%----------HEADING-----------------

\begin{center}\textbf{\Large Егор Бодунов }\end{center}
\vspace{-12pt}
\begin{center} \MYhref{mailto:egorbodunoff@yandex.ru }{ \faEnvelope \space egorbodunoff@yandex.ru} \quad  \MYhref{ https://t.me/bmp_header }{ \faPaperPlane \space bmpheader } \quad \MYhref{ https://github.com/egorbodunoff }{ \faGithub \space egorbodunoff } \quad  
\end{center}


  

%-----------SUMMARY-----------------



  

%-----------EXPERIENCE-----------------

\vspace{-5pt}
\section{Опыт работы}
\justifying
  \resumeSubHeadingListStart
        \vspace{30pt}
        % \resumeExpSubheading
%           {\MYhref{ https://ozon.ru }{ Ozon\ExternalLink }}
%           {  }
%           { Стажер аналитик }
%           { Фераль 2024 - Present }
%           {  \skill{ Airflow }  \skill{ SQL }  \skill{ python }  \skill{ SuperSet }  }
%           \resumeDesc{
%           \begin{itemize}
              
%                 \item Автоматизировала комплексные отчеты для стратегического анализа, прототипизировала и оптимизировала скрипты на SQL и python и непрерывно интегрировала в систему, используя связку GitLab+AF.
              
%                 \item Проводила исследования по аномалиям, вследствие чего разрабатывала новую методологию и писала документацию для кросс-командного взаимодействия.
              
%                 \item Создавала дэшборды в BI системе SuperSet, для которых подготавливала датасеты в ClickHouse.
%                 % -- закрытие отчетного периода - сбор данных для P\&L в части направления бизнеса
%                 % -- cash\&stock management - сбор данных по будущим и прошлым платежам
%                 % -- прототипирование запросов, непрерывная интеграция (CI) новых фич, автоматизация развертываемых отчетов в окружении (связка GitLab+AF)
%                 % -- Создавала дэшборды в BI системе SuperSet, для которых подготавливала датасеты в ClickHouse
              
%           \end{itemize}}

%     % \resumeExpSubheading
%     %       {\MYhref{ https://ozon.ru }{ Ozon\ExternalLink }}
%           \vspace{5pt}
%           \resumeExpSubheading
%           {\MYhref{ https://www.gningi.ru}{ 
%           % Государственный научно-исследовательский навигационно-гидрографический институт
%           АО ГНИГИ
%           \ExternalLink }}
%           { }
%           { Младший научный сотрудник }
%           { Декабрь 2021 - Декабрь 2023}
%           {  \skill{ C++ }  \skill{ MathCad }  \skill{ ranges }  \skill{ thread }   }
%           \resumeDesc{
%           \begin{itemize}
              
%                 \item Спроектировала математические модели для навигации в MathCad, применяя знания аналитической геомерии и теоретической механики.
              
%                 \item Спроектировала компьютерные модели для реализации математических, применив знания ООП, шаблонов проектирования и
%                 Для оптимизации я использовала параллельные вычисления и ranges.
              
%                 \item С лета 2023 была главной в команде: ставила встречи, наставляла и помогала в решение задач коллег.\\
              
%           \end{itemize}}
  
  \resumeSubHeadingListEnd



  

%-----------PROJECTS-----------------

\section{Проекты}
  \resumeSubHeadingListStart
  
      % \resumeProjSubheading
      % {\MYhref{ https://github.com/data-bases-team/back }{ QRCodeMenu \ExternalLink}}{}
      % {  \skill{ Django } \skill{ Docker } \skill{ SQL } \skill{ html/css } \skill{ nginx } }
      %     \resumeDesc{
      %     \begin{itemize}
            
      %         \item Инструмент для учета, редактирования и просмотра позиций меню, сканируя QR код
            
      %         \item Все изменения обновляются в реальном времени, легко добавить или удалить позицию
            
      %         \item Возможность быстрой развертки нового экземпляра меню для нового заведения (мультитенантность за счет контейнеров Docker)
            
      %         \item Автоматический учет длительности жизни каждого контейнера
              

            
      %     \end{itemize}
      %     }
  
      \resumeProjSubheading
      {\MYhref{ https://education.vk.company/program/172 }{ Deep Python \ExternalLink}}{VK education}
      {  \skill{ Python }  \skill{ Asincio } \skill{ Threading } \skill{ Git } }
          \resumeDesc{
          \begin{itemize}
            
              \item В качетве выпуского проекта написали в команде многопоточный фреймворк на питоне --- аналог Django.
            
              % \item Implemented abstract syntax tree traversal for finding suitable constants.
            
              % \item Used enumarr at work for maintaining relevance of state machine tests.
              
            
          \end{itemize}
          }

        \resumeProjSubheading
      {\MYhref{ https://education.vk.company/program/280 }{ Основы машинного обучения \ExternalLink}}{ VK education x МГТУ}
      {  \skill{ Python }  \skill{ Sklearn } \skill{ NLTK }  }
          \resumeDesc{
          \begin{itemize}
            
              \item В качестве домашних заданий участвовал в большом количестве соревнований по ML на платформе vk\_cups.
              \item Rлассификации контента 18+ на наборе данных, содержащих url и title веб-страниц.
              \item Кластеризация sparse матрицы\\
              % \item Implemented abstract syntax tree traversal for finding suitable constants.
            
              % \item Used enumarr at work for maintaining relevance of state machine tests.
            
          \end{itemize}
          }
  
  \resumeSubHeadingListEnd



  

%-----------EDUCATION-----------------

\section{Образовние}
  \resumeSubHeadingListStart
    
    \resumeEducationSubheading
    { \MYhref{ https://bmstu.ru }{ МГТУ им. Баумана }}{ Москва, Россия }
    { Биомедицинские технические системы }
    { Сентябрь 2021 - Present } \\\\
    \textbf{Предметы:} линейная алгебра, дискретная математика, ООП, базы данных, ТВиМС, ОС, медицинские информационные системы, микропроцессорные системы.\\

    \resumeEducationSubheading
    { \MYhref{ https://education.vk.company/program/280 }{ Основы машинного обучения }}{ Москва, Россия }
    { VK education x МГТУ }{ Февраль 2024 - Present } \\\\
    \textbf{Предметы:} введение в ML, алгоритмы и структуры данных, системный дизайн, Python. \\

    \resumeEducationSubheading
    {\MYhref{ https://education.vk.company/program/172 }{ Deep Python }}{Москва, Россия}
    {VK education}{ Сентябрь 2023 - Декабрь 2023 } \\\\

    \resumeEducationSubheading
    {\MYhref{ https://education.vk.company/program/217 }{ Введение в анализ данных }}{Москва, Россия}
    {VK education}{ Сентябрь 2023 - Октябрь 2023 } \\\\
% \textbf{Факултативы}: волонтерская деятельность, организация English Speaking Club

    % \\
    % \\
    % \\\\
    

 % Units included: Linear algebra, Discrete math, Object-Oriented Programming, Databases, Probability theory and mathematical statistics, OS.
      
    
  \resumeSubHeadingListEnd

  

%--------PROGRAMMING SKILLS------------

\section{Навыки}
 % \resumeSubHeadingListStart
 \begin{tabular}{ll}
 \textbf{Языки:} & \quad Python, SQL, C/C++, bash
 \\ 
 \textbf{Библиотеки:} & \quad Pandas, Numpy, Scipy, Sklearn, pytorch, NLTK, Matplotlib, Asyncio, Threading \\ 
 \textbf{Инструменты:} & \quad Git, Docker, Excel, Vertica\\ 
\end{tabular}
 % \resumeSubHeadingListEnd




\end{document}



















